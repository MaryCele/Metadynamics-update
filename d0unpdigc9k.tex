System we are investigating: b2m wild type, d76n, w60g.

Why these systems and which are the characteristics?

Chosen technique: well tempered metadynamics

Definition of collective variables

Simulation of monomer in solutions: length of more than 1microsecond, and observing the deposition of the gaussian for the two collective variables we haven't reached the convergence yet. So we need to continue while doing tools for the analyses.

Analyses we are performing:
observation of the free energy landscape 


MATERIAL: the studied systems and the main questions
Among all the amino acids that are the basics of protein composition, Proline is the most unusual one: it has a cyclic side chain that is bound to the amide nitrogen of the backbone  

[\cite{assembly} -]
The isomerisation of Pro32 from its native cis to the non-native trans configuration is indicated as a trigger factor for b2m misfolding and the subsequent amyloidogenical mechanism.
Cis-configuration for the Proline 32 (the other 4 proline are in trans configuration) is required in order to maintain the soluble native state of B2M wild type, while trans-configuration is required for amyloid elongation at neutral pH. From 10 years ago work, [Eichner, Radford - Volume 386, Issue 5, 13 March 2009, Pages 1312-1326 ] we know that assembly is shown to involve the transient formation of a non-native monomer containing a trans backbone conformation. subsequently there is the formation of dimeric species and higher aggregates that accumulate before the development of amyloid fibrils. (This is the milestone for Radford and co to say that Pro32 isomerisation

In this work, fluorinated proline derivatives are used as probe to investigate the structure-function relationships in b2m. Such substitution alters the equilbrium population of trans and cis isomers via stereoelectronics effects and also lowers the barrier for the isomerisation.
Previous studies showed that trans to cis prolyl bond isomerisation of Pro32 is the rate-limiting step in the protein folding mechanism.

Experimental works suggested that fibril formation occurs via metastable, partially unfolded protein conformers. These species are obtained when Pro32, starting from the normal cis-conformation, slowly adapts to trans geometry, leading to destabilizing effects on the structure of the protein: in particular there is an exposition of the hydrophobic structure and also a rearrangement of the D strand that lead to intermolecular aggregation. 